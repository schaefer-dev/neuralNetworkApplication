% exercise sheet with header on every page for math or close subjects
\documentclass[12pt]{article}
\usepackage[utf8]{inputenc} 
\usepackage{latexsym} 
\usepackage{multicol}
\usepackage{fancyhdr}
\usepackage{amsfonts} 
\usepackage{amsmath}
\usepackage{amssymb}
\usepackage{enumerate}
\usepackage{listings}
\usepackage{graphicx}

% Shortcuts for bb, frak and cal letters
\newcommand{\E}{\mathbb{E}}
\newcommand{\V}{\mathbb{V}}
\renewcommand{\P}{\mathbb{P}}
\newcommand{\N}{\mathbb{N}}
\newcommand{\R}{\mathbb{R}}
\newcommand{\C}{\mathbb{C}}
\newcommand{\Z}{\mathbb{Z}}
\newcommand{\Pfrak}{\mathfrak{P}}
\newcommand{\Pfrac}{\mathfrak{P}}
\newcommand{\Bfrac}{\mathfrak{P}}
\newcommand{\Bfrak}{\mathfrak{B}}
\newcommand{\Fcal}{\mathcal{F}}
\newcommand{\Ycal}{\mathcal{Y}}
\newcommand{\Bcal}{\mathcal{B}}
\newcommand{\Acal}{\mathcal{A}}

% formating
\topmargin -1.5cm 
\textheight 24cm
\textwidth 16.0 cm 
\oddsidemargin -0.1cm

% Fancy Header on every Page
\pagestyle{fancy}
\lhead{\textbf{Pattern and Speech Recognition}}
\rhead{Daniel Schäfer (2549458)\\ Christian Bohnenberger (2548364) \\ Dominik Weber (2548553)}
\renewcommand{\headrulewidth}{1.2pt}

\setlength{\headheight}{45pt} 

\begin{document}
\pagenumbering{gobble}

\section*{1 Linear Algebra}
\begin{enumerate}[1)]
    \item
        M symetrie $\Rightarrow M=VDV^{-1}$,the decomposition with V orthogonal.\\
        $$\frac{x^TMx}{\Vert x_2^2\Vert}=\frac{<x,Mx>}{\Vert x\Vert_2^2}=\frac{<x,VDV^{-1}x>}{\Vert x\Vert_2^2}=\frac{<x,VDV^Tx>}{\Vert x\Vert_2^2}=\frac{<V^Tx,DV^Tx>}{\Vert x\Vert_2^2}=$$$$\ldots \text{substidute y}:=V^Tx\ldots=\frac{<y,Dy>}{\Vert Vy\Vert_2^2}=\frac{<y,Dy>}{<Vy,Vy>}=\frac{<y,Dy>}{<y,V^TVy>}=$$$$\frac{<y,Dy>}{<y,y>}=\frac{\sum_{i=1}^{n}\lambda_iy_i^2}{\Vert y\Vert_2^2}=*$$\\$$\frac{\lambda_{min}\sum_{i=1}^{n}y_i^2}{\Vert y\Vert_2^2}\leq * \leq \frac{\lambda_{max}\sum_{i=1}^{n}y_i^2}{\Vert y\Vert_2^2}\Rightarrow \lambda_{min}\frac{\Vert y \Vert_2^2}{\Vert y\Vert_2^2}\leq * \leq \lambda_{max}\frac{\Vert y \Vert_2^2}{\Vert y\Vert_2^2} \Rightarrow \lambda_{min} \leq * \leq \lambda_{max}$$
    \item
        a real symmetric matrix A can be decomposed as: $A=QEQ^T$ ,where Q is an orthogonal matrix and E is a diagonal matrix whose entries are the eigenvalues of A.\\
    $\Rightarrow A^{100}=(QEQ^T)^{100}$\\
    with Associative property and $QQ^T = I \Rightarrow A^{100}=QE^{100}Q^{T}$\\
    $E^{100}$ is easy to compute, because you just need to compute $\lambda_1^{100},\lambda_2^{100},\lambda_3^{100},\ldots,\lambda_n^{100}$.\\

    \item
        \begin{enumerate}[(i)]
            \item 
                % TODO
                $$det\begin{bmatrix}2&-1&-1\\-1&3&-1\\-1&-1&x\end{bmatrix}=$$\\
                $2*3*x+(-1)*(-1)*(-1)+(-1)*(-1)*(-1)-(-1)*(3)*(-1)-2*(-1)*(-1)-(-1)*(-1)*x = 5x-7 $\\
                $\Rightarrow 5x-7 > 0 \rightarrow x> \frac{7}{5}$\\
                For every value $ x \geq \frac{7}{5}$ the matrix is positive definite because the det. is greater than 0. E.g., $x=2$ is a possible value to get a positive definite matrix.
            \item
                % TODO
               	 $$\begin{bmatrix}2&-1\\-1&3\end{bmatrix}* \begin{bmatrix} u \\v \end{bmatrix} = \begin{bmatrix} -1 \\-1 \end{bmatrix}$$\\
               	 \begin{enumerate}
               	 	\item[I] $2u-v=-1 \rightarrow v=2u+1$
               	 	\item[II] $-u+3v= -1 $
               	 \end{enumerate}
               	 $v=2u+1$ in $II$:\\
               	 $-u+3*(2u+1)=-1 \rightarrow u=-\frac{4}{5}$\\
               	 $\rightarrow v=-\frac{3}{5}$\\
               	 $x=-1*(-\frac{4}{5})+(-1)*(-\frac{3}{5}) = \frac{7}{5}$\\
               	 That means if we set $x=\frac{7}{5}$ and we add the first row of the matrix times $-\frac{4}{5}$ and the second row times $-\frac{3}{5}$ we last will be (0,0,0).
        \end{enumerate}

    \item
        \begin{enumerate}[(i)]
            \item 
                y=2\\$$\begin{bmatrix}2&-1&-1\\-1&2&-1\\-5&3&y\end{bmatrix}*\begin{bmatrix}1\\1\\1\end{bmatrix}=\begin{bmatrix}0\\0\\0\end{bmatrix}$$
                the equation is solved when $-5+3+y=0$ is solved.
            \item
                y=-4\\
                The eigenvalues are the solutions for the equation $$det\begin{bmatrix}2-\lambda&-1&-1\\-1&2-\lambda&-1\\-5&3&y-\lambda\end{bmatrix}=0$$\\$\Rightarrow-\lambda^3+4*\lambda^2-\lambda+\lambda^2*y-4*\lambda*y+3*y-6=0$\\
                The eigenvalues are \\
                $\lambda_{1} = 3$\\
                $\lambda_{2} = (1+y-\sqrt{9-2y+y^2})/2$\\
                $\lambda_{3} = (1+y+\sqrt{9-2y+y^2})/2$\\
                The sum of the eigenvalues is 0 if y=-4\\
        \end{enumerate}
\end{enumerate}

\newpage
\section*{2 Probability Theory}
\begin{enumerate}[1)]
    \item 
        % TODO
        $$ P(A)= \frac{\mid \{2,4,6\} \mid}{\mid \{1,2,3,4,5,6\} \mid} = \frac{1}{2}$$ 
        $$ P(B)= \frac{\mid \{1,2,3,4\} \mid}{\mid \{1,2,3,4,5,6\} \mid} = \frac{2}{3}$$ 
        $$ P(C)= \frac{\mid \{1,3,5\} \mid}{\mid \{1,2,3,4,5,6\} \mid} = \frac{1}{2}$$ 
        
        $$ P(A \cap B)= \frac{\mid \{2,4\} \mid}{\mid \{1,2,3,4,5,6\} \mid} = \frac{1}{3}$$ 
        $$ P(A) * P(B) = \frac{1}{2} * \frac{2}{3} = \frac{1}{3} $$
        $ \Rightarrow $ A and B are independent
        
        $$ P(A \cap C)= \frac{\mid \{\} \mid}{\mid \{1,2,3,4,5,6\} \mid} = 0$$ 
        $$ P(A) * P(C) = \frac{1}{2} * \frac{1}{2} = \frac{1}{2} $$
        $ \Rightarrow $ A and C are not independent
        
    \item
        $$ P(\text{Macintosh}) = 30\% $$
        $$ P(\text{Windows})= 50\% $$
        $$ P(\text{Linux})= 20\% $$

        $$ P( \text{Virus } \vert \text{ Macintosh}) = 65\% $$
        $$ P( \text{Virus } \vert \text{ Windows}) = 82\% $$
        $$ P( \text{Virus } \vert \text{ Linux}) = 50\% $$

        and we are looking for $P(\text{Windows } \vert \text{ Virus}) = \text{?}$

        $$ P( \text{Virus}) = P( \text{Virus } \vert \text{ Macintosh}) * P( \text{Macintosh}) $$
        $$+ P( \text{Virus } \vert \text{ Windows}) * P( \text{Windows}) $$
        $$+ P( \text{Virus } \vert \text{ Linux}) * P( \text{Linux}) $$
        $$ = 0,65*0,3 + 0,82*0,5 + 0,5 * 0,2 = \underline{0,705} $$

        $$ P(A \vert B) = \frac{P(B \vert A) * P(A)}{P(B)} $$

        $$ P( \text{Virus } \vert \text{ Windows}) = \frac{ P( \text{Windows } \vert \text{ Virus}) * P( \text{Virus})}{P( \text{Windows})} $$
        $$ 0,82 = \frac{ P( \text{Windows } \vert \text{ Virus}) * 0,705}{0,5}$$
        $$ P( \text{Windows } \vert \text{ Virus}) = \underline{0,581560284} $$

    \item
        If $f$ and $g$ are PDFs the following conditions have to hold:
        \begin{itemize}
            \item 
                the integral from $-\infty$ to $\infty$ has to be 1
            \item
                every value $f(x)$ / $g(x)$ has to be $\geq 0$
        \end{itemize}

        we check these conditions for $f$ and $g$:
        \begin{itemize}
            \item 
                For $f$ we compute $\int_{-\infty}^{\infty} f(x) = \int_{0}^{\infty} \frac{1}{1+x} (+0)$. This inegral does not converge so $f(x)$ can not be a PDF!
            \item
                For $g$ we compute $\int_{-\infty}^{\infty} g(x) = \int_{0}^{\infty} \frac{1}{(1+x)^2} (+0)$. This integral is $ = 1$ and thus $g$ fullfills the first condition.

                The second condition also holds, because $0 \geq 0$ and $\forall x \quad \frac{1}{(1+x)^2} \geq 0$

                \underline{$\Rightarrow$ $g(x)$ is a valid probability density function!}

        \end{itemize}


        % TODO
        \textbf{the mean of $g$:}
        $$\text{mean } = E(X) = \int_0^{\infty} x g(x) dx = \underline{ 0 }$$ 

    \item
        % TODO
        no solution

    \item
        \textbf{Proof:}\\
        $$ E( X Y ) = \sum_x \sum_y x * y * P(X=x, Y=y) $$
        $$ = \sum_x \sum_y x * y * P(X=x) * P(Y=y)$$
        $$ = \sum_x  x * P(X=x) * \sum_y y * P(Y=y)$$
        $$ = E(X) * E(Y)$$

        % TODO unsicher
        \textbf{Covariance:}\\
        $$ cov(X,Y) = E[(X - E[X]) * (Y - E[Y])] $$
        $$ = E[XY - X E[Y] - E[X]*Y + E[X]E[Y]]$$
        $$ = E[XY] - E[X]E[Y] - E[X]E[Y] + E[X]E[Y]$$
        $$ = E[XY] - E[X]E[Y] = 0$$

                
\end{enumerate}



\newpage
\section*{3 Multivariable Calculus}
 
        % TODO
        $$ \nabla f = \begin{bmatrix}3x^2-3y^2\\-6xy\end{bmatrix} $$
        $$ Hess f(x) = \begin{bmatrix} 6x & -6y \\ -6y & -6x \end{bmatrix} $$
        
        $$ \nabla f = 0 $$
         
        \begin{enumerate}
         	\item[I:] $ 3x^2-3y^2=0 \Leftrightarrow x^2-y^2=0 \Leftrightarrow x^2=y^2 \Leftrightarrow x=y \vee x=-y $
         	\item[II:] $ -6xy=0 \Leftrightarrow x*y=0 $
        \end{enumerate}
        
        $x=y$ in II:\\
        $$ y*y = 0 \Rightarrow y = 0 \rightarrow x = 0 $$
        
		$x=-y$ in II:\\
        $$ (-y)*y = 0 \Rightarrow y = 0 \rightarrow x = 0 $$
        
        possible extermal points: (0,0)\\
        
        $$ Hess f(0,0) = \begin{bmatrix} 0 & 0 \\ 0 & 0 \end{bmatrix} $$
        
        $$ det (Hess f(0,0)) = 0 $$
        
        Eigenvalues of Hess f(0,0): $\lambda_0=0$ and $\lambda_1=0$\\
        
        Both Eigenvalues are 0, i.e., the matrix is positive semi-definite and negative semi-definite. \\
        
        $\Rightarrow$ There is a direction whose eigenvalue is semi-positive and another direction whose Eigenvalue is semi-negative $\Rightarrow$ graph has a saddle at (0,0)\\
        
        Since there is no local max. or min. we do not have to check if there is a local point which is also a global min. or max.\\
        
        Maximum and Minimum value:\\
        
        $$ lim_{(x,y)\rightarrow (\infty,0)}f(x)= \infty $$
        
        $$ lim_{(x,y)\rightarrow (-\infty,0)}f(x)= -\infty $$
        
        The function has min. value: $-\infty$ and max. value: $\infty$ (which are also global min./max.)
        

\end{document}
